%From Evaluation of Definite Integral sin^m cos^n dx from 0 to 2pi for Even Number n,m.tex
%Copyright ⓒ 2015  박승원(seungwonpark), Gyeonggi Science HighSchool for the Gifted. All rights reserved. Email : psw14041@gmail.com
% Provided by GSHS LaTeX Intro.
% Website : http://gshslatexintro.github.io
\documentclass[11pt]{article}
\usepackage[left=25mm,right=25mm,top=30mm,bottom=30mm]{geometry}
\usepackage{amsmath} % math
\usepackage{amssymb} % math
\usepackage{amsthm}
\usepackage{graphicx} % to use \includegraphics{}
\usepackage{diagbox} % to make tables
\usepackage{kotex}
\usepackage{enumitem}
\usepackage[hidelinks]{hyperref}
\usepackage{fancyhdr}
\pagestyle{fancy}
\cfoot{\tiny{Copyright ⓒ 2015  박승원(seungwonpark), Gyeonggi Science HighSchool for the Gifted. All rights reserved. Email : psw14041@gmail.com}}
\newtheorem{definition}{정의}
\newtheorem{lemma}{보조정리}
\newtheorem{gap}{비약}
\begin{document}
\begin{center}
	\Large Evaluation of $\int_{0}^{2\pi}{\sin^m x \cos^n xdx}$ for Even Number n,m
\end{center}
\begin{flushright}
	2015.9.6 by 14041 박승원
\end{flushright}
\tableofcontents

\section{Introduction}

우리는 미적분학 I 시간에 삼각함수로 이루어진 다양한 함수의 적분 방법에 대해서 알아보았다. 그 중에서 나의 흥미를 끈 것은 $m,n$이 짝수일때 $\int{\sin^m x \cos^n xdx}$ 의 적분 방법이었다.
\footnote{반각 공식 $\sin^2 x = \frac{1-\cos2x}{2}$, $\cos^2 x = \frac{1+\cos 2x}{2}$를 통해 적분식을 점점 간단하게 만드는 방법이다.} 하지만 이 방법은 깔끔하지 못했고, $\int{\sin^m x \cos^n xdx}$의 일반화된 부정적분 식을 알아보고자 노력하였지만, 그것은 분명히 어려운 일이었다. 따라서 나는 정적분 $\int_{0}^{2\pi}{\sin^m x \cos^n xdx}$ 이라도 구해 보았다.

\begin{definition}
	For $2|m$ and $2|n$,
	
	$$ A(m,n) = \int_{0}^{2\pi}{\sin^m x \cos^n xdx} $$
\end{definition}

\section{By getting/using recursion formula} \label{recursion}
우리는 우선 부분적분을 통하여 $A(m,n)$과 다른 어떤 식과의 관계식, 즉 점화식을 구해 볼 것이다.
\begin{equation} \label{partial_integration}
\int_{0}^{2\pi}{\sin^m x \cos^n xdx} = \int_{0}^{2\pi}{\sin^m x \cos^{n-1} x \cdot \cos x dx}
\end{equation}


$\sin^m x\cos^{n-1} x$를 미분하면
$m\sin^{m-1}\cos^n x - (n-1)\sin^{m+1}x\cos^{n-2}x$이고,
$\cos x$를 적분하면 $\sin x$이므로 식 \ref{partial_integration}을 정리하면 식 \ref{recursion_formula}과 같다.

\begin{equation} \label{recursion_formula}
A(m,n)=\frac{n-1}{m+1}A(m+2,n-2)
\end{equation}

$\therefore A(m,n)=\frac{(n-1)(n-3)\cdots\cdot(1)}{(m+1)(m+3)\cdots(m+n-1)}A(m+n,0)$

일반화된 형태로 $A(m,n)$을 나타내면 식 \ref{A}와 같다.
\begin{equation} \label{A}
A(m,n)=A(m+n,0)\cdot\frac{\frac{n!}{2^{n/2}\cdot(n/2)!}\frac{m!}{2^{m/2}\cdot(m/2)!}}{\frac{(m+n)!}{2^{(m+n)/2}\cdot{(m+n)/2}!}}
\end{equation}

\begin{definition}
	For $2|k$,
	
	$$B(k)=\int_{0}^{2\pi}{\sin^k xdx}$$
\end{definition}

$B(k)$ 역시  부분적분을 통해 점화식을 구할 수 있다. 

\begin{equation} \label{partial_integration-2}
\int_{0}^{2\pi}{\sin^k xdx} = \int_{0}{2\pi}{\sin^{k-1}x \cdot \sin x dx}
\end{equation}

$\sin^{k-1}{x}$를 미분하면 $(k-1)\sin^{k-2}{x}\cos{x}$이고,
$\sin{x}$를 적분하면 $-\cos{x}$이므로
식 \ref{partial_integration-2}는 식 \ref{recursion_formula-2}와 같이 정리된다.

\begin{equation} \label{recursion_formula-2}
B(k)=\frac{k-1}{k}B(k-2)
\end{equation}
$\therefore B(k)=\frac{(k-1)(k-3)\cdots(1)}{k(k-2)\cdots\cdot2}B(0)$

$B(0)=2\pi$ 등을 고려하여 $B(k)$를 일반화시키면 식 \ref{B}와 같다.
\begin{equation} \label{B}
B(k)=2\pi \cdot \frac{k!}{(2^{k/2}\cdot(k/2)!)^2}
\end{equation}

식 \ref{A}와 식 \ref{B}를 종합하면 결과적으로 식 \ref{recursion_result}와 같다.

\begin{equation} \label{recursion_result}
A(m,n)=2\pi \cdot \frac{\frac{n!}{2^{n/2}\cdot(n/2)!}\frac{m!}{2^{m/2}\cdot(m/2)!}}{2^{(m+n)/2}\cdot((m+n)/2)!}
\end{equation}


\section{By using Euler's equation} \label{complex}
두번째 풀이 방법은 삼각함수를 복소수를 이용하여 나타내고 그를 적분하는 방법이다. 이 방법에서 한가지 비약이 있음에 유의해야 한다. 식 \ref{complex_integration}와 같이 지수함수의 적분을 임의로 복소수에까지 확장했다는 점이다. 일단은 허용하고 넘어가자.
\footnote{복소함수 $f(z)=u+iv$ 가 미분가능하기 위해서는 Cauchy-Riemann 관계식 
$\frac{\partial u}{\partial x} = \frac{\partial v}{\partial y};
 \frac{\partial v}{\partial x} = -\frac{\partial u}{\partial y}
$을 만족하면 된다고 한다. 더 알아보고 싶은 사람은
\href{http://www.math.columbia.edu/~rf/complex2.pdf}{"Complex Functions and the Cauchy-Riemann Equations"}을 참고하길 바란다. (나도 잘 모름)
}

\begin{gap} \label{complex_integration}
For real number $x$ and complex number $\alpha$ , 
$$\int{e^{\alpha x}} = \frac{e^{\alpha x}}{\alpha} + C $$
\end{gap}

아무튼 우리는 식 \ref{complex_trigonometrics}과 보조정리 \ref{complex_integration_idea} 을 핵심적으로 사용할 것이다.

\begin{equation} \label{complex_trigonometrics}
\cos x = \frac{e^{ix}+e^{-ix}}{2} , \sin x = \frac{e^{ix}-e^{-ix}}{2i}
\end{equation}

\begin{lemma} \label{complex_integration_idea}
For non-zero integer m, $\int_{0}^{2\pi}{e^{imx}} = [\frac{e^{imx}}{im}]_{0}^{2\pi} = 0$
\end{lemma}

A(m,n)을 식 \ref{complex_integration}을 이용하여 전개하자. \emph{단, 간편한 계산을 위해 일반성을 잃지 않고 $m\geq n$이라 가정하자.}

$$A(m,n)=\int_{0}^{2\pi}{ (\frac{e^{ix}-e^{-ix}}{2i})^m \cdot (\frac{e^{ix}+e^{-ix}}{2})^n dx }$$.

피적분함수의 항들 중에서 상수항이 아닌 것은 보조정리 \ref{complex_integration_idea}에 의해 모두 0이 되고, 상수항만이 남는다. 상수항을 계산하면 식 \ref{complex_summation}와 같다.

\begin{equation} \label{complex_summation}
\frac{i^m}{2^{m+n}} \cdot \sum_{k=0}^{n}{_{n}C_k \cdot (-1)^k \cdot _{m}C_k}
\end{equation}

$A(m,n)$은 이 상수항을 적분한 값인 $2\pi \cdot constant$가 되고, 결론적으로 $A(m,n)$은 식 \ref{result_complex}와 같다.

\begin{equation} \label{result_complex}
A(m,n)=2\pi \cdot \frac{i^m}{2^{m+n}} \cdot \sum_{k=0}^{n}{_{n}C_k \cdot (-1)^k \cdot _{m}C_k}
\end{equation}

\section{Verification}
$m=2, n=2$인 경우에 대해 일반적인 방법, \ref{recursion}절의 방법, \ref{complex}절의 방법을 통해 계산해보고 서로 일치하는지 확인하자.

\begin{enumerate}
	\item 일반적인 방법 : 
	
	$A(2,2) = \int_{0}^{2\pi}{\sin^2{x}\cos^2{x}dx}
	= \int_{0}^{2\pi}{\frac{\sin^2{2x}}{4}dx}
	= \int_{0}^{2\pi}{\frac{1-\cos{4x}}{8}dx}
	= [\frac{x}{8}-\frac{\sin{4x}}{32}]_{0}^{2\pi}
	= \frac{\pi}{4}
	$
	
	\item \ref{recursion}절의 방법 :
	
	$A(2,2) = 2\pi \cdot \frac{\frac{2!}{2^{2/2}\cdot(2/2)!}\frac{2!}{2^{2/2}\cdot(2/2)!}}{2^{4/2}\cdot(4/2)!}
	= \frac{\pi}{4}
	$
	
	\item \ref{complex}절의 방법 : 
	
	$A(2,2) = 2\pi \cdot \frac{i^2}{2^4} \cdot \sum_{k=0}^{2}{_{2}C_k \cdot (-1)^k \cdot _{2}C_k}
	= \frac{\pi}{4}
	$
	
\end{enumerate}


\section{Further Research}
우선은 \ref{recursion}절의 방법과 \ref{complex}절의 방법이 어떻게 같은 결과가 나오는 것인지 알아보아야 할 것이다. 다시 말하면, 다음 식이 성립함을 증명해야 한다.

For even number n and m,  
$$
\frac{\frac{n!}{2^{n/2}\cdot(n/2)!}\frac{m!}{2^{m/2}\cdot(m/2)!}}{2^{(m+n)/2}\cdot((m+n)/2)!} = \frac{i^m}{2^{m+n}} \cdot \sum_{k=0}^{n}{_{n}C_k \cdot (-1)^k \cdot _{m}C_k}
$$

또한, 이외에도 정적분 A(m,n)을 구하는 방법을 더 찾아보고 서로 비교하여 새로운 등식을 얻게 된다면 좋을 것이다.


\end{document}
